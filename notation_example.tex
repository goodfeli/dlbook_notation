\newif\ifColor
% Add \Colortrue to settings.tex to use color
% Otherwise, will show a primarily grayscale version of the document for printing

\input{settings.tex}

\documentclass[11pt,oneside,a4paper]{book}
\usepackage{natbib}
\usepackage{breakcites} % Do Not let citations break out of the text frame
\usepackage{microtype} % Get rid of some frame busts automatically
% Note: if microtype causes error on ubuntu, run
% sudo apt-get install cm-super
\title{Example Notation for Deep Learning}
\author{Ian Goodfellow\\Yoshua Bengio\\Aaron Courville}
\date{}
\setcounter{tocdepth}{1}

\pdfobjcompresslevel=0

\usepackage{zref-abspage}

\setcounter{secnumdepth}{3} % Number subsubsections, because we reference them,
% so the reader needs numbers to find the correct place.


\usepackage[vcentering,dvips]{geometry}
\geometry{papersize={7in,9in},bottom=3pc,top=5pc,left=5pc,right=5pc,bmargin=4.5pc,footskip=18pt,headsep=25pt}


%%% Packages %%%
\usepackage{epsfig}
\usepackage{subfigure}
\usepackage[utf8]{inputenc}

% Needed for some foreign characters
\usepackage[T1]{fontenc}

\usepackage{amsmath}
\usepackage{subfigure}
\usepackage{amsfonts}
\usepackage{amsthm}
\usepackage{multirow}
\usepackage{colortbl}
\usepackage{booktabs}
% This allows us to cite chapters by name, which was useful for making the
% acknowledgements page
\usepackage{nameref}
% Make sure there is a space between the subsection number and subsection title
% in the table of contents.
% If we do not do this we end up with 2 digit subsection numbers colliding with
% the title.
% See https://tex.stackexchange.com/questions/7853/toc-text-numbers-alignment/7856#7856?newreg=d2632892dd0345f388619f12fa794b11
\usepackage[tocindentauto]{tocstyle}
\usetocstyle{standard}

\usepackage{bm}


\usepackage{float}
\newcommand{\boldindex}[1]{\textbf{\hyperpage{#1}}}
\usepackage{makeidx}\makeindex
% Make bibliography and index appear in table of contents
\usepackage[nottoc]{tocbibind}
% Using the caption package allows us to support captions that contain "itemize" environments.
% The font=small option makes the text of captions smaller than the main text.
\usepackage[font=small]{caption}

% Used to change header sizes
\usepackage{fancyhdr}



\usepackage[chapter]{algorithm}
\usepackage{algorithmic}
% Include chapter number in algorithm number
\renewcommand{\thealgorithm}{\arabic{chapter}.\arabic{algorithm}}


\theoremstyle{definition}
\newtheorem{example}{Example}[section]

% Define the P table cell environment
% It is the same as p, but centers the text horizontally
\usepackage{array}
\newcolumntype{P}[1]{>{\centering\arraybackslash}p{#1}}

% Rebuild the book document class's headers from scratch, but with different font size
% (this is for MIT Press style)
% Source: http://texblog.org/2012/10/09/changing-the-font-size-in-fancyhdr/
\newcommand{\changefont}{% 1st arg to fontsize is font size. 2nd arg is the baseline skip. both in points.
    \fontsize{9}{11}\selectfont
}
\fancyhf{}
%\fancyhead[LE,RO]{\changefont \slshape \rightmark} %section
\fancyhead[RE,LO]{\changefont \slshape \leftmark} %chapter
\fancyfoot[C]{\changefont \thepage} %footer
\pagestyle{fancy}
\input{math_commands.tex}

\usepackage[pdffitwindow=false,
pdfview=FitH,
pdfstartview=FitH,
pagebackref=true,
breaklinks=true,
\ifColor
colorlinks,
\fi
bookmarks=false,
plainpages=false]{hyperref}

% Make \[ \] math have equation numbers
\DeclareRobustCommand{\[}{\begin{equation}}
\DeclareRobustCommand{\]}{\end{equation}}

% Allow align environments to cross page boundaries.
% If we do not do this, we get weird gaps of several inches of white space
% before or after some long align environments.
\allowdisplaybreaks

\begin{document}

\setlength{\parskip}{0.25 \baselineskip}
\newlength{\figwidth}
\setlength{\figwidth}{26pc}
% Spacing between notation sections
\newlength{\notationgap}
\setlength{\notationgap}{1pc}

\typeout{START_CHAPTER "TOC" \theabspage}
\frontmatter


\maketitle
\tableofcontents
\typeout{END_CHAPTER "TOC" \theabspage}

 \chapter*{Notation}
\label{notation}


\typeout{START_CHAPTER "notation" \theabspage}

% Sometimes we have to include the following line to get this section
% included in the Table of Contents despite being a chapter*
\addcontentsline{toc}{chapter}{Notation}
This section provides a concise reference describing notation used throughout this
document.
If you are unfamiliar with any of the corresponding mathematical concepts,
\citet{dlbook} describe most of these ideas in
chapters 2--4.

\vspace{\notationgap}
% Need to use minipage to keep title of table on same page as table
\begin{minipage}{\textwidth}
% This is a hack to put a little title over the table
% We cannot use "\section*", etc., they appear in the table of contents.
% tocdepth does not work on this chapter.
\centerline{\bf Numbers and Arrays}
\bgroup
% The \arraystretch definition here increases the space between rows in the table,
% so that \displaystyle math has more vertical space.
\def\arraystretch{1.5}
\begin{tabular}{cp{3.25in}}
$\displaystyle a$ & A scalar (integer or real)\\
$\displaystyle \va$ & A vector\\
$\displaystyle \mA$ & A matrix\\
$\displaystyle \tA$ & A tensor\\
$\displaystyle \mI_n$ & Identity matrix with $n$ rows and $n$ columns\\
$\displaystyle \mI$ & Identity matrix with dimensionality implied by context\\
$\displaystyle \ve^{(i)}$ & Standard basis vector $[0,\dots,0,1,0,\dots,0]$ with a 1 at position $i$\\
$\displaystyle \text{diag}(\va)$ & A square, diagonal matrix with diagonal entries given by $\va$\\
$\displaystyle \ra$ & A scalar random variable\\
$\displaystyle \rva$ & A vector-valued random variable\\
$\displaystyle \rmA$ & A matrix-valued random variable\\
\end{tabular}
\egroup
\index{Scalar}
\index{Vector}
\index{Matrix}
\index{Tensor}
\end{minipage}

\vspace{\notationgap}
\begin{minipage}{\textwidth}
\centerline{\bf Sets and Graphs}
\bgroup
\def\arraystretch{1.5}
\begin{tabular}{cp{3.25in}}
$\displaystyle \sA$ & A set\\
$\displaystyle \R$ & The set of real numbers \\
% NOTE: do not use \R^+, because it is ambiguous whether:
% - It includes 0
% - It includes only real numbers, or also infinity.
% We usually do not include infinity, so we may explicitly write
% [0, \infty) to include 0
% (0, \infty) to not include 0
$\displaystyle \{0, 1\}$ & The set containing 0 and 1 \\
$\displaystyle \{0, 1, \dots, n \}$ & The set of all integers between $0$ and $n$\\
$\displaystyle [a, b]$ & The real interval including $a$ and $b$\\
$\displaystyle (a, b]$ & The real interval excluding $a$ but including $b$\\
$\displaystyle \sA \backslash \sB$ & Set subtraction, i.e., the set containing the elements of $\sA$ that are not in $\sB$\\
$\displaystyle \gG$ & A graph\\
$\displaystyle \parents_\gG(\ervx_i)$ & The parents of $\ervx_i$ in $\gG$
\end{tabular}
\egroup
\index{Scalar}
\index{Vector}
\index{Matrix}
\index{Tensor}
\index{Graph}
\index{Set}
\end{minipage}

\vspace{\notationgap}
\begin{minipage}{\textwidth}
\centerline{\bf Indexing}
\bgroup
\def\arraystretch{1.5}
\begin{tabular}{cp{3.25in}}
$\displaystyle \eva_i$ & Element $i$ of vector $\va$, with indexing starting at 1 \\
$\displaystyle \eva_{-i}$ & All elements of vector $\va$ except for element $i$ \\
$\displaystyle \emA_{i,j}$ & Element $i, j$ of matrix $\mA$ \\
$\displaystyle \mA_{i, :}$ & Row $i$ of matrix $\mA$ \\
$\displaystyle \mA_{:, i}$ & Column $i$ of matrix $\mA$ \\
$\displaystyle \etA_{i, j, k}$ & Element $(i, j, k)$ of a 3-D tensor $\tA$\\
$\displaystyle \tA_{:, :, i}$ & 2-D slice of a 3-D tensor\\
$\displaystyle \erva_i$ & Element $i$ of the random vector $\rva$ \\
\end{tabular}
\egroup
\end{minipage}

\vspace{\notationgap}
\begin{minipage}{\textwidth}
\centerline{\bf Linear Algebra Operations}
\bgroup
\def\arraystretch{1.5}
\begin{tabular}{cp{3.25in}}
$\displaystyle \mA^\top$ & Transpose of matrix $\mA$ \\
$\displaystyle \mA^+$ & Moore-Penrose pseudoinverse of $\mA$\\
$\displaystyle \mA \odot \mB $ & Element-wise (Hadamard) product of $\mA$ and $\mB$ \\
% Wikipedia uses \circ for element-wise multiplication but this could be confused with function composition
$\displaystyle \mathrm{det}(\mA)$ & Determinant of $\mA$ \\
\end{tabular}
\egroup
\index{Transpose}
\index{Element-wise product|see {Hadamard product}}
\index{Hadamard product}
\index{Determinant}
\end{minipage}

\vspace{\notationgap}
\begin{minipage}{\textwidth}
\centerline{\bf Calculus}
\bgroup
\def\arraystretch{1.5}
\begin{tabular}{cp{3.25in}}
% NOTE: the [2ex] on the next line adds extra height to that row of the table.
% Without that command, the fraction on the first line is too tall and collides
% with the fraction on the second line.
$\displaystyle\frac{d y} {d x}$ & Derivative of $y$ with respect to $x$\\ [2ex]
$\displaystyle \frac{\partial y} {\partial x} $ & Partial derivative of $y$ with respect to $x$ \\
$\displaystyle \nabla_\vx y $ & Gradient of $y$ with respect to $\vx$ \\
$\displaystyle \nabla_\mX y $ & Matrix derivatives of $y$ with respect to $\mX$ \\
$\displaystyle \nabla_\tX y $ & Tensor containing derivatives of $y$ with respect to $\tX$ \\
$\displaystyle \frac{\partial f}{\partial \vx} $ & Jacobian matrix $\mJ \in \R^{m\times n}$ of $f: \R^n \rightarrow \R^m$\\
$\displaystyle \nabla_\vx^2 f(\vx)\text{ or }\mH( f)(\vx)$ & The Hessian matrix of $f$ at input point $\vx$\\
$\displaystyle \int f(\vx) d\vx $ & Definite integral over the entire domain of $\vx$ \\
$\displaystyle \int_\sS f(\vx) d\vx$ & Definite integral with respect to $\vx$ over the set $\sS$ \\
\end{tabular}
\egroup
\index{Derivative}
\index{Integral}
\index{Jacobian matrix}
\index{Hessian matrix}
\end{minipage}

\vspace{\notationgap}
\begin{minipage}{\textwidth}
\centerline{\bf Probability and Information Theory}
\bgroup
\def\arraystretch{1.5}
\begin{tabular}{cp{3.25in}}
$\displaystyle \ra \bot \rb$ & The random variables $\ra$ and $\rb$ are independent\\
$\displaystyle \ra \bot \rb \mid \rc $ & They are conditionally independent given $\rc$\\
$\displaystyle P(\ra)$ & A probability distribution over a discrete variable\\
$\displaystyle p(\ra)$ & A probability distribution over a continuous variable, or over
a variable whose type has not been specified\\
$\displaystyle \ra \sim P$ & Random variable $\ra$ has distribution $P$\\% so thing on left of \sim should always be a random variable, with name beginning with \r
$\displaystyle  \E_{\rx\sim P} [ f(x) ]\text{ or } \E f(x)$ & Expectation of $f(x)$ with respect to $P(\rx)$ \\
$\displaystyle \Var(f(x)) $ &  Variance of $f(x)$ under $P(\rx)$ \\
$\displaystyle \Cov(f(x),g(x)) $ & Covariance of $f(x)$ and $g(x)$ under $P(\rx)$\\
$\displaystyle H(\rx) $ & Shannon entropy of the random variable $\rx$\\
$\displaystyle \KL ( P \Vert Q ) $ & Kullback-Leibler divergence of P and Q \\
$\displaystyle \mathcal{N} ( \vx ; \vmu , \mSigma)$ & Gaussian distribution %
over $\vx$ with mean $\vmu$ and covariance $\mSigma$ \\
\end{tabular}
\egroup
\index{Independence}
\index{Conditional independence}
\index{Variance}
\index{Covariance}
\index{Kullback-Leibler divergence}
\index{Shannon entropy}
\end{minipage}

\vspace{\notationgap}
\begin{minipage}{\textwidth}
\centerline{\bf Functions}
\bgroup
\def\arraystretch{1.5}
\begin{tabular}{cp{3.25in}}
$\displaystyle f: \sA \rightarrow \sB$ & The function $f$ with domain $\sA$ and range $\sB$\\
$\displaystyle f \circ g $ & Composition of the functions $f$ and $g$ \\
  $\displaystyle f(\vx ; \vtheta) $ & A function of $\vx$ parametrized by $\vtheta$.
  (Sometimes we write $f(\vx)$ and omit the argument $\vtheta$ to lighten notation) \\
$\displaystyle \log x$ & Natural logarithm of $x$ \\
$\displaystyle \sigma(x)$ & Logistic sigmoid, $\displaystyle \frac{1} {1 + \exp(-x)}$ \\
$\displaystyle \zeta(x)$ & Softplus, $\log(1 + \exp(x))$ \\
$\displaystyle || \vx ||_p $ & $\normlp$ norm of $\vx$ \\
$\displaystyle || \vx || $ & $\normltwo$ norm of $\vx$ \\
$\displaystyle x^+$ & Positive part of $x$, i.e., $\max(0,x)$\\
$\displaystyle \1_\mathrm{condition}$ & is 1 if the condition is true, 0 otherwise\\
\end{tabular}
\egroup
\index{Sigmoid}
\index{Softplus}
\index{Norm}
\end{minipage}

Sometimes we use a function $f$ whose argument is a scalar but apply
it to a vector, matrix, or tensor: $f(\vx)$, $f(\mX)$, or $f(\tX)$.
This denotes the application of $f$ to the
array element-wise. For example, if $\tC = \sigma(\tX)$, then $\etC_{i,j,k} = \sigma(\etX_{i,j,k})$
for all valid values of $i$, $j$ and $k$.


\vspace{\notationgap}
\begin{minipage}{\textwidth}
\centerline{\bf Datasets and Distributions}
\bgroup
\def\arraystretch{1.5}
\begin{tabular}{cp{3.25in}}
$\displaystyle \pdata$ & The data generating distribution\\
$\displaystyle \ptrain$ & The empirical distribution defined by the training set\\
$\displaystyle \sX$ & A set of training examples\\
$\displaystyle \vx^{(i)}$ & The $i$-th example (input) from a dataset\\
$\displaystyle y^{(i)}\text{ or }\vy^{(i)}$ & The target associated with $\vx^{(i)}$ for supervised learning\\
$\displaystyle \mX$ & The $m \times n$ matrix with input example $\vx^{(i)}$ in row $\mX_{i,:}$\\
\end{tabular}
\egroup
\end{minipage}

\clearpage

\typeout{END_CHAPTER "notation" \theabspage}

 \mainmatter
 \chapter{Commentary}
\label{chap:commentary}
\typeout{START_CHAPTER "intro" \theabspage}

This document is an example of how to use the accompanying files
as well as some commentary on them.
The files are {\tt math\_commands.tex} and {\tt notation.tex}.
The file {\tt math\_commands.tex} includes several useful {\LaTeX}
macros and {\tt notation.tex } defines a notation page that could
be used at the front of any publication.

We developed these files while writing \citet{dlbook}.
We release these files for anyone to use freely, in order to help
establish some standard notation in the deep learning community.


\section{Examples}
\label{sec:examples}

We include this section as an example of some {\LaTeX} commands
and the macros we created for the book.

Citations that support a sentence without actually being used in the sentence
should appear at the end of the sentence using {\tt citep}:

\begin{quote}
Inventors have long dreamed of creating machines that think.
This desire dates back to at least the time of ancient Greece.
The mythical figures Pygmalion, Daedalus, and Hephaestus may
all be interpreted as legendary inventors, and
Galatea, Talos, and Pandora may all be regarded as artificial
life \citep{ovid2004metamorphoses,sparkes1996red,1997works}.
\end{quote}

When the authors of a document or the document itself are a noun in the
sentence, use the {\tt citet} command:

\begin{quote}
\citet{Mitchell:1997:ML} provides a succinct definition of machine learning:
``A computer program is said to learn from experience $E$ with respect to some
class of tasks $T$ and performance measure $P$, if its performance at tasks in
$T$, as measured by $P$, improves with experience $E$.''
\end{quote}

When introducing a new term, using the {\tt newterm} macro to highlight it.
If there is a corresponding acronym, put the acronym in parentheses
afterward. If your document includes an index, also use the {\tt index}
command.

\begin{quote}
Today, \newterm{artificial intelligence} (AI)\index{Artificial intelligence} is
a thriving field with many practical applications and active research topics.
\end{quote}

Sometimes you may want to make many entries in the index that all point
to a canonical index entry:

\begin{quote}
One of the simplest
and most common kinds of parameter norm penalty is
the squared $\normltwo$ parameter norm penalty
commonly known as \newterm{weight decay}.
\index{Weight decay}
In other academic communities,
$\normltwo$ regularization is also known as \newterm{ridge regression}
or \newterm{Tikhonov regularization}.
\index{Ridge regression|see {weight decay}}\index{Tikhonov regularization|see {weight decay}}
\end{quote}

To refer to a figure, use either {\tt figref} or {\tt Figref} depending on
whether you want to capitalize the resulting word in the sentence.

\begin{quote}
See \figref{fig:venn} for an example of a how to include graphics
in your document.
\Figref{fig:venn} shows how to include graphics in your document.
\end{quote}


\begin{figure}[t!]
\centering
\includegraphics{venn}
\caption{An example of a figure.
The figure is a PDF displayed without being rescaled within {\LaTeX}.
The PDF was created at the right size to fit on the page, with the
fonts at the size they should be displayed. The fonts in the figure
are from the Computer Modern family so they match the fonts used
by \LaTeX.}
\label{fig:venn}
\end{figure}

Similarly, you can refer to different sections of the book using
{\tt partref}, {\tt Partref}, {\tt secref}, {\tt Secref}, etc.

\begin{quote}
	You are currently reading \secref{sec:examples}.
\end{quote}

\section*{Acknowledgments}
We thank Catherine Olsson and \'Ulfar Erlingsson for proofreading and
review of this manuscript.


\clearpage
%%
\typeout{END_CHAPTER "intro" \theabspage}


\small{
\typeout{START_CHAPTER "bib" \theabspage}
\bibliography{notation}
\bibliographystyle{natbib}
\clearpage
\typeout{END_CHAPTER "bib" \theabspage}
}
\typeout{START_CHAPTER "index-" \theabspage}
\printindex
%\clearpage
\typeout{END_CHAPTER "index-" \theabspage}
%\newpage


\end{document}
